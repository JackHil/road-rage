\documentclass[letterpaper, 10 pt, conference]{ieeeconf}  % Comment this line out
                                                          % if you need a4paper
%\documentclass[a4paper, 10pt, conference]{ieeeconf}      % Use this line for a4
                                                          % paper

\IEEEoverridecommandlockouts                              % This command is only
                                                          % needed if you want to
                                                          % use the \thanks command
\overrideIEEEmargins
% See the \addtolength command later in the file to balance the column lengths
% on the last page of the document

\usepackage[pdftex]{graphicx}
% TODO remember we need to take out all the color
\newcommand{\pix}[3]{
  \begin{figure}[h]
    \centering \includegraphics[#3]{#1}
    \caption{#2}
  \end{figure}
}

% TODO better name!
\title{\LARGE \bf
Approximately Orchestrated Road Traffic Automata\\
a General Traffic Simulation Framework
}

\author{Dustin Carlino, Mike Depinet, Piyush Khandelwal, and Peter Stone\\
        Department of Computer Science\\
        The University of Texas at Austin\\
        Austin, TX 78712\\
        {\tt \small\{dcarlino,msd775,piyushk,pstone\}@cs.utexas.edu}}

\long\def\commentp#1{{\bf **Peter: #1**}}
\long\def\commentpk#1{{\bf **Piyush: #1**}}
\long\def\commentm#1{{\bf **Mike: #1**}}
\long\def\commentd#1{{\bf **Dustin: #1**}}

% \long\def\commentp#1{}
% \long\def\commentpk#1{}
% \long\def\commentm#1{}
% \long\def\commentd#1{}

\begin{document}

\maketitle
\thispagestyle{empty}
\pagestyle{empty}

%%%%%%%%%%%%%%%%%%%%%%%%%%%%%%%%%%%%%%%%%%%%%%%%%%%%%%%%%%%%%%%%%%%%%%%%%%%%%%%%

\commentd{sounds fine to me, only thing is that we haven't actually focused on
  dynamic re-planning or any kind of overseer yet, although theres nothing
  stopping us from trying that idea again}
\commentp{If it's not implemented, we shouldn't talk about it.}

\begin{abstract} 
Autonomous vehicles have seen great advancements in recent years and
such vehicles are now ever closer to being commercially available.
The advent of driver-less cars provides opportunities for optimizing
traffic in ways not possible before. This paper introduces an open
source multi-agent traffic simulator called AORTA, which stands for
\textit{Approximately Orchestrated Road Traffic Automata}, and
demonstrates its usage for optimizing autonomous traffic at a
city-wide scale. AORTA creates scale simulations by generating maps
using publicly available road data from \textit{OpenStreetMap}
(OSM). Simulations of specific regions throughout the world can be set
up in the few minutes it takes to export data for the region of
interest from OSM. AORTA focuses on interactions between agents by
defining \textit{behaviors} for individual vehicles and
intersections. These behaviors allow agents to interact with each
other as well as with an external \textit{overseer} that is
responsible for optimizing traffic flow. Overseers can interact with
these agents to allow for dynamic route re-planning to reduce traffic
load, optimized intersection navigation, and improved car following.
\commentpk{rest devoted to application}
\end{abstract}

%%%%%%%%%%%%%%%%%%%%%%%%%%%%%%%%%%%%%%%%%%%%%%%%%%%%%%%%%%%%%%%%%%%%%%%%%%%%%%%%

\section{INTRODUCTION}
\label{sec:introduction}

% theme: easy simulation in your city, in 2 minutes!
% be sure to emphasize open source!
% contribution: autonomous car / human car sim on EXISTING INFRASTRUCTURE

Autonomous vehicle technology has made tremendous progress in the last decade.
In 2004, the farthest distance traveled by a vehicle autonomously in the DARPA
Grand Challenge was $11.9km$ \cite{cnnGrandChallenge2004}. By 2007, six of the
competing teams completed the $96km$ course set for the DARPA Urban Challenge
\cite{spectrumUrbanChallenge2007}. They did so while following the same traffic
laws followed by human drivers, navigating along with other moving vehicles,
and following correct intersection precedence order. Since then, Google's
driver-less cars have clocked more than $250,000km$ on public roads in urban
California, USA \cite{tedThrun2011}. In 2010, researchers from the University
of Parma successfully completed an autonomous \textit{intercontinental} run
from Parma, Italy to Shanghai, China \cite{cnnVislab2010}. The successful
completion of all these milestones suggests that autonomous cars are here to
stay, and are ever closer to becoming commercially available. 

The arrival of driver-less cars presents several challenges that need to be
addressed in the near term. Road infrastructure needs to be updated to better
support autonomous vehicles.\commentp{Is this necessarily the case?
  It may be useful, but is it a need?} A number of legal and safety concerns need to be
addressed in the context of driver-less cars. Human drivers need new
methodologies to interact with autonomous drivers on public roads. At the same
time, driver-less cars provide opportunities of solving traffic problems in
ways that were never before possible. For instance, autonomous cars can be
quickly diverted to make way for emergency vehicles responding to an accident.
Even day-to-day traffic issues such as congestion could potentially be greatly
reduced by effective management of traffic flow. This work addresses
some of these problems.\commentp{Be more specific in this paragraph.
  Which of these challenges are eing addressed elsewhere (cite), and
  which are addressed in this paper?}
\commentp{Overall, this paragraph is a bit scattered.  I suggest
  something more like the following (after the first sentence).
  Several of the challenges are already being addressed.  For example
  .....
  This paper focusses on the specific challenge(s) of....}

\commentd{nevada even made autonomous cars legal:
http://www.pcmag.com/article2/0,2817,2400400,00.asp}

\commentd{another example long-term usage: fuel. I went to a talk by
  somebody on the engr side about powermiling and optimizing battery usage}

\commentd{we don't support lane-changing yet, and overseers arent re-implemented
yet either}

\commentp{legal issues are addressed here:  http://www.path.berkeley.edu/PATH/Publications/PDF/PRR/2009/PRR-2009-28.pdf}

\pix{downtown_atx.png}{A simulation of downtown Austin, Texas.}{scale=0.25}

This paper introduces an open source multi-agent microscopic traffic
simulator called AORTA, which stands for \textit{Approximately
Orchestrated Road Traffic Automata}. The AORTA simulator focuses on
the interactions between individual vehicles and intersections, both
of which act as \textit{agents} in AORTA's framework. Vehicles
interact with other vehicles to pass and change lanes, and interact
with intersections to decide when to cross them. Researchers
developing applications on top of the AORTA simulator define
\textit{behaviors} that these agents follow to interact with one
another. To optimize traffic flow, external \textit{overseers} can
coordinate with these agents to provide recommendations such as route
suggestions or intersection signaling policies.  An advantage of such
an approach is that by assigning human-like behaviors such as the
``car following model'' \cite{brackstone1999car} to vehicles, and
traffic-signal-like policies to intersection, AORTA can just as easily
simulate human traffic.

Like any other microscopic traffic simulator, AORTA needs maps to run
simulations. One of AORTA's key features is that it generates maps using road data available from OpenStreetMap
(OSM) \cite{osm}, a Wikipedia-like interface for world maps. Researchers using
AORTA can export the relevant map area from OSM that is then parsed by AORTA to
create scale simulations of the real world. Using real world road data is
advantageous for a number of reasons. First, researchers using AORTA for human
traffic simulations would find it advantageous to also have the ability to use
real-world roads. Second, even as infrastructural changes will be made to the
road infrastructure to accommodate driver-less cars, it is reasonable to assume
that the \textit{same} road network will still be in use.\commentp{I
  don't understand this second point} Third, using
real-world data from OSM allows setting up simulations for an area of interest
in a matter of minutes, thereby allowing easy transfer of research results to
any desired region.

\commentd{The first reason for real world data almost feels too obvious to state
like that}
\commentp{I think all 3 reasons are similar.  I'd just say simplty
  that AORTA's key features are being open source, easy to install and
  get running, and flexibility to quickly and easily incorporate
  models from real road networks around the world.}

\commentpk{Needs 1 more para based on final application}

% Section 2 will review other traffic simulators. Section 3 will introduce the
% overall architecture of AORTA. Section 4 describes how OSM maps are used.
% Section 5 explains the simulation engine and the properties AORTA posses that
% are useful for general traffic experiments. Section 6 gives some initial
% results from such an experiment, and Section 7 concludes.

%%%%%%%%%%%%%%%%%%%%%%%%%%%%%%%%%%%%%%%%%%%%%%%%%%%%%%%%%%%%%%%%%%%%%%%%%%%%%%%%

\section{RELATED WORK}
\label{sec:related_work}

Computational processing power has made excellent advancements in the
last two decades. Parallel computing has proved significant for
traffic simulation through the use of Geographic Information System
(GIS) software \cite{pursula1999simulation}. Advanced computing
techniques have enabled microscopic models of traffic simulation to
generate results at a meaningful scale (city-wide or greater). As a
result, a number of good micro-simulators have been introduced in the
past decade. We review a number of such simulators in this section
along with other relevant work.

A number of traffic simulators already make use of OSM data. MATSim is one such
multiagent simulator that focuses on performing large-scale simulations in a
relatively small amount of time \cite{balmer2009matsim}. Traffic demand is
supplied to MATSim in the form of \textit{plans} that an individual may intend
to follow in a given day. MATSim aims to iteratively improve these plans though
offline computation to maximize traffic throughput. In other words, it helps
individuals understand the best way to go about their daily routine to minimize
time spent in transit. This goal is somewhat different from that of AORTA,
where the demand supplied by individuals is taken as is. Instead, AORTA aims to
reduce traffic congestion through interactions with individuals in real-time
based on the current demand.

Another popular open source traffic simulator that can use OSM data is SUMO
\cite{SUMO2011}, which shares the same goals as AORTA and is fairly similar at
a high level. SUMO has been used to study the effect of automated
transportation systems, route planning of individual vehicles, as well as
dynamically adapting traffic signal policies to increase traffic efficiency.
AORTA and SUMO differ in the manner in which intersections are handled. SUMO
uses traditional traffic signals, while AORTA adopts a much more general
\textit{reservation} system. A reservation tells a car when it is safe to take
its desired path through the intersection. Reservations can be allotted to
vehicles through more traditional means such as traffic signals or stop-sign
based intersection precedence, or through more efficient methods designed
specifically for autonomous vehicles \cite{JAIR08-dresner}.
%\commentpk{does the above difference seem a bit pedantic?}  

On the other hand, there are a number of simulators that specifically
target autonomous traffic systems. One such system is AIM
\cite{JAIR08-dresner}, which aims to optimize traffic flow between
autonomous vehicles at a given intersection. AIM studies problems at
the scale of roughly a dozen intersections arranged in a regular grid,
while AORTA has the potential to apply some of the same research at a
city-wide scale to real-world road networks. A few simulators focus on
autonomous vehicles with sensors and actuators
\cite{figueiredo2009approach}. These simulators are used to ensure
that autonomous vehicles are behaving correctly while interacting with
other vehicles and infrastructure. On the other hand, AORTA assumes
that autonomous vehicles are capable of reasonable driving, and
focuses on interactions between agents to optimize traffic flow.

Autonomous agents have also been used in the past to model human
traffic. For instance, one approach models human merging and lane
changing behavior through the use of autonomous agents interacting
with one another \cite{hidas2002modelling}. Another approach attempted
to improve traffic congestion for human drivers by using a network of
autonomous intersections.  These intersections have the ability to
interact with one another and execute a dynamic traffic signal policy
minimizing overall wait time \cite{manikonda2001autonomous}.

\commentd{and we can do the above too. we talk about how to make the current
  behavior more flawed and humanlike in the agent reactions section.}

%%%%%%%%%%%%%%%%%%%%%%%%%%%%%%%%%%%%%%%%%%%%%%%%%%%%%%%%%%%%%%%%%%%%%%%%%%%%%%%%

\section{ARCHITECTURE}

\commentd{consider re-grouping sections... maybe this is all 'methodology'?}

\pix{architecture.png}
    {An illustration of AORTA's components.}
    {scale=0.2}

AORTA is divided into three modular components: the map model, a
micro-simulation engine, and a UI. The map model transforms OSM maps into AORTA
graphs, then answers pathfinding and geometry queries. The simulation engine
adds a notion of agents, vehicle dynamics, and collisions. A headless mode
enables running experiments without the overhead of visualization. Finally, the
UI, which uses Swing, can interactively render the map and agents. For
convenience, specific traffic patterns can be constructed between two regions of
the map by simply drawing a polygon around each set of roads.

AORTA's implementation uses Scala, a language implemented on the Java Virtual
Machine. Scala provides the advantage of functional programming constructs while
still permitting imperative style. Parts of AORTA, including extensions and clients, can
just as easily be written in Java. The software has a documented API, is
open-source\footnote{http://code.google.com/p/road-rage}, and contains all of
the code for the experiments described in Section 6 -- it is designed for
extensibility.

%%%%%%%%%%%%%%%%%%%%%%%%%%%%%%%%%%%%%%%%%%%%%%%%%%%%%%%%%%%%%%%%%%%%%%%%%%%%%%%%

\section{MAP CONSTRUCTION FROM OSM}

Our goal is to simulate traffic on existing infrastructure, not contrived,
randomly generated maps. We do this by utilizing OpenStreetMaps, a community
project offering user-editable, free maps. To run a simulation on a map from
OSM, there is a one-time conversion process to transform the map into AORTA's
format, which is also encoded with XML.

\subsection{Map Model}

\pix{map_model.png}
    {Roads contain several lanes, and intersections contain several turns
     between lanes.}
    {scale=0.5}

\commentd{is this clear now? it's not a multigraph and there are no loops, since
  edges are lanes and turns. I have to think a bit about intersections as
  vertices strictly, though...}

A map is represented as a directed graph with \emph{lanes} and \emph{turns} as edges and
\emph{intersections} as vertices. A \emph{road} groups one or two groups of
lanes together, each group being a different direction of travel. Intersections
map the possible turns from incoming to outgoing lanes. Since both lanes and
turns support traffic, they are grouped together as the \emph{traversables} or
edges, meaning they have some total length.

The map also has a geometric interpretation, used for visualization and physical
distances. A road has a sequence of points giving the line segments of the
center yellow line. These points could also be interpreted as control points for
a curved road, but AORTA uses straight lines as an approximation. Lanes are
parallel to the center line, some width apart from one another.  Turns are
modeled with a single straight line connecting the end of one edge to the start
of another (so right turns often have $0$ length). 

\subsection{Map Construction Passes}

\commentd{gotta work on passive voice}

Map construction is split into several passes, each feeding the next.
\begin{enumerate}
  \item OSM encodes many paths besides driveable roads, so these are first
        filtered out. GPS locations common to multiple roads are marked, since
        they implicitly indicate intersections.
  \item OSM \emph{ways} (sequences of GPS locations) include many
        intersections, so the builder next splits ways into undirected
        segments of roads between exactly two intersections.
  \item Undirected roads are multiplied into directed lanes in each direction
        (unless the road is marked one-way). The number of lanes is guessed from
        OSM's ``road type'' tag or an explicit number of lanes, when given.
  \item Turns are next constructed between incoming and outgoing edges at each
        vertex. Left, right, straight, and U-turns are heuristically chosen
        based on the angles between edges. When several lanes all cross into
        fewer lanes, merging is arbitrarily forced on the rightmost lanes.
        Although inaccurate, these patterns are the best compromise. At the
        time of writing, even sources like Google Maps (TODO cite) lack this
        information.
  \item Because turns are heuristically placed, the graph is disconnected. The
        strongly connected components of the digraph with the fewest members are
        removed. Tarjan's algorithm locates these in linear time (TODO cite).
        Disconnected lanes are spliced out of larger roads, leaving unappealing
        geometric gaps, but correctly ensuring connectivity for corner-case free
        pathfinding.
  \item The builder serializes the graph in a simple XML format.
\end{enumerate}

%%%%%%%%%%%%%%%%%%%%%%%%%%%%%%%%%%%%%%%%%%%%%%%%%%%%%%%%%%%%%%%%%%%%%%%%%%%%%%%%

\section{SIMULATION}

AORTA uses microscopic simulation, modeling individual drivers as agents.
Discrete-time simulation is used, meaning an agent accelerates at a fixed rate
for the duration of a ``tick'' (also referred to as ``step'') to achieve a new
position and velocity. The tick duration $dt$ is fixed, enabling agents to
reason about collision avoidance and satisfying other constraints. Setting this
value to the average ``reaction time'' of a human driver may be appropriate for
some experiments. Each tick, the simulation performs the following steps:

\begin{enumerate}
  \item Introducing new agents into the map
  \item Updating the position and velocity of agents based on their choice the
        previous tick
  \item Checking for collisions
  \item Allowing each agent to react
\end{enumerate}

Agents have a few enforced constraints -- they cannot exceed their physical
acceleration capabilities, travel in reverse, or travel past the end of a
traversable without appearing on the next. Intersection policies and avoiding
collisions on the same traversable provide the other constraints, but it is the
job of an agent's behavior to satisfy those. There are three main components of
the simulation, each configurable by extending default implementations of an
interface: agent behaviors, routing strategies, and intersection policies.

%\commentd{justify arbitrary constants, like deaccel for cars;
%          systems ideas, or limitations.. trylocks, atomic sim update, deadlock
%          issues;
%          limit: no lane-changing}

\subsection{Spawning}

To introduce new agents into the simulation in real-time, users create
\emph{generators} programatically or using the UI. A generator picks a random
starting position and random goal from two sets of roads chosen by the user,
selectable by drawing a polygon with the mouse in the UI. A generator may cover
the entire map, uniformly distributing traffic everywhere, or it can mimic
rush-hour scenarios.

Every tick, a generator creates some number of new agents. If the agent's route
policy needs work to be done (such as pathfinding), the generator either
performs it immediately (incurring a delay in simulation if it is running) or
sends it to a pool of worker threads to compute in the background.

Once an agent's route is ready, the agent waits alongside its starting edge (as
if it is in a driveway or parking lot). When the edge has no agents that could
potentially crash with the new vehicle, the new agent enters the system with
zero initial speed.  Because behaviors must enforce speed limit along an edge by
the time an agent enters the road, then if there is no agent within the
worst-case distance (traveling distance at the speed limit, plus stopping
distance from this speed) before the spawn point, then the new agent may enter.
To render unnecessary the prediction of which agents could enter the edge in the
near future, there is a buffer space at the beginning of the edge where a new
agent may never spawn. This disqualifies some short edges from being spawn
points, but this is not a big limitation in practice.

\subsection{Collision checking}

All traversables (edges and the straight line-approximation of turns) maintain
an ordered queue of occupying agents. Agents stepping forward in time happens
atomically, so collision checking occurs after all agents have moved for some
tick. This is done because otherwise false positives could occur when an agent
steps before the agent in front of it, and ordering agents for updates is
difficult near the crossings between edges. \commentd{Do I need to explain why
it is difficult, or should I just stop after ``in front of it''?} A collision is
defined as the reversal of order of agents before and after a step.

Likewise, intersections check for collisions by examining the turn each agent is
performing and verifying that no other agent is simultaneously performing a
conflicting turn.  AORTA has a low-granularity model of turn conflict. If two
vehicles at any point along two different turns could ever come into contact,
then the turns are always in conflict, no matter where the agent specifically is
along that turn.  The alternative of tiling intersections \cite{JAIR08-dresner}
would provide more granularity, but it adds computational complexity and would
be impractical for many of the complex intersection geometries inferred from
OSM.

When a collision occurs, the simulator throws warnings and halts, since this
indicates a bug in some policy. A future alternative could be to mark the region
as impassable and re-route other traffic away from the accident.

\pix{turn_conflicts.png}
    {An agent cannot turn left while agents on the perpendicular road cross}
    {scale=0.5}

\subsection{Agent Reactions}

Each agent has a \emph{behavior} governing it. When the agent travels past the end
of an edge during its step, the behavior picks the turn to pursue. Each tick,
the behavior can make the agent react by performing one of two actions:
disappearing from the map (only when the agent is at rest and is done with its
route) or setting an acceleration for the next step. Were lane changing or
merging modeled, that would be another possible action.

Currently, the primary behavior used by all agents is the ``route-following
behavior.'' It is a generalized, baseline behavior that guarantees not to
collide with another agent or enter an intersection at the wrong time. Since it
picks the highest safe choice of acceleration at each step, it could be easily
extended to mimic human drivers by traveling at some random speed slower than
this or to optimize fuel efficiency by tuning acceleration.

\subsubsection{Worst-case analysis and lookahead}

\commentd{one diagram probably covers lots of this}

The default behavior maximizes acceleration while obeying speed limits,
avoiding collisions, and obeying intersections. By considering the highest speed
achievable at the end of the next tick and the subsequent stopping distance at
this speed, the behavior can scan sufficiently far ahead along several
traversables to locate agents to avoid and intersections to query. The analysis
can short-circuit at the first nearby agent and intersection requiring a stop,
since it would be illegal to pass either.

\subsubsection{Interaction with intersections}

% TODO regroup? does this belong as a subsection, not subsub?

A behavior's interaction with an intersection policy amounts to polling it with
the turn the agent wants to perform, the distance away the agent currently is,
and how long the agent has been waiting (so policies can enforce a required
delay before entering the intersection). The intersection policy simply orders
the agent to ``stop'' or ``continue.'' The acceleration to avoid entering an intersection
attempts to end a configurable threshold back from the end of a traversable to
account for crosswalks.

Intersection policies are one of the most intriguing aspects of the simulation
to tweak, especially in light of autonomous vehicles. Current policies include
traditional stop signs and traffic signals. Future policies could include stop
lights that adaptively learn better timings and an AIM-based reservation system
\cite{JAIR08-dresner}. 

\commentd{Actually, at least the stop sign policy is really mundane. Include its
  description?}

%The stop sign policy works by maintaining a current owner and a queue of waiting
%agents. An agent can enter the queue only when it is within some threshold of
%the start of its turn. After the agent at the head of the queue has waited for a
%configurable pause, it obtains the lock and has exclusive access to the
%intersection, releasing it only when it completes its turn.

\commentd{describe traffic signal policy, once it's working fully}

\subsection{Routing Strategies}

Picking accelerations that optimize some objective functions is a separate task
from picking a path to travel. Behaviors consult a route strategy to pick turns.
Currently two simple implementations are available. One statically plans a route
to some goal using standard A* search (TODO cite); the other performs a drunken
walk, lazily planning random choices at each intersection as the lookahead
engine needs to know. Future experiments in dynamic replanning or hierarchial
planning can be performed without modifying any code other than a creating new
route implementation. Route policies could also interact with a singleton
overseer to have some sort of global insight.

\subsection{Deadlock}

Maintaining following distance from agents actually enables a form of deadlock
in the simulation. If an agent is forced to stop in an intersection mid-turn due
to its destination edge filling up with other agents, then nobody may cross the
blocked intersection. With many agents in a small area, this situation may
occur several times in interdependent ways. The solution currently being
investigated will require behaviors to ensure a turn may be fully completed
before starting it. This will prevent intersections from ever being blocked.

%\commentp{Cite (and ideally use ideas from) paper commented out in latex file
    % @inproceedings{AAAI11-au,
    % Author = {Tsz-Chiu Au and Neda Shahidi and Peter Stone},
    % Booktitle= {Proceedings of the Twenty-Fifth Conference on Artificial Intelligence},
    % Title = {Enforcing Liveness in Autonomous Traffic Management},
    % Year = {2011},
    % Month= {August},
%}

% TODO solve the above case, then figure out what to describe.

%A form of deadlock is possible in a sequence of short edges and turns. Agent
%$A_1$ acquires the lock to intersection $V_1$, but needs the lock to $V_2$ to
%proceed. Meanwhile, $A_2$ has the lock for $V_2$ and needs the lock for $V_1$.
%To deal with this circumstance, an agent's behavior can yield any held locks (a
%notion generalized to any intersection policy) when its speed is $0$ and its
%conservative acceleration is to remain at rest. The stop sign policy only
%removes an agent's lock if it has not started its turn (because otherwise, the
%agent is stalling in the intersection). The next time that agent polls, the
%policy restores the agent to the front of the queue.

%\commentd{hopefully we'll have a neater solution than the above soon}

\subsection{Determinism and other simulation properties}

AORTA is designed for controlled experiments. Consequently, simulations are
deterministically reproducible. A single seed for the pseudo-random number
generator, the input graph, and the generators' configuration completely
determine the outcome of any simulation. This determinism does break down when
generators use worker threads, since routes may be computed in different orders,
causing agents to be ready to enter the road at different times.

\commentd{saving a ``scenario'' can soon be a feature}

\commentd{and how will we solve this issue with threads and determinism?}

\commentd{give some informal or formal (with numbers) idea of scale / how many
agents we can simulate in different map sizes} \commentpk{I would say that the the last point is fairly important}

%%%%%%%%%%%%%%%%%%%%%%%%%%%%%%%%%%%%%%%%%%%%%%%%%%%%%%%%%%%%%%%%%%%%%%%%%%%%%%%%

\section{EXPERIMENTAL RESULTS}

\commentd{upgrading stop signs to signals, optimizing timing?}

%%%%%%%%%%%%%%%%%%%%%%%%%%%%%%%%%%%%%%%%%%%%%%%%%%%%%%%%%%%%%%%%%%%%%%%%%%%%%%%%

\section{CONCLUSIONS}

Write me. big future ideas:

\commentp{Proposed improvements to OSM data would be great for a
  discussion or conclusion section at the end of the paper.}
proposals to change OSM to fix these issues
  projecting lines sucks... but all it really does is clutter UI
  filtering out invalid ways in pass 1 is really not a clear process
  number of lanes and speed limits... road types are wacky and wrong
  turns are brittle.. detecting right-turn only lane is rare. shared center lane?
    not part of our model, or OSM's.
  removing sccs is not ideal

contraflow

microtolling

dynamic replanning

autonomous intersections

\addtolength{\textheight}{-12cm}  % This command serves to balance the column lengths
                                  % on the last page of the document manually. It shortens
                                  % the textheight of the last page by a suitable amount.
                                  % This command does not take effect until the next page
                                  % so it should come on the page before the last. Make
                                  % sure that you do not shorten the textheight too much.

%%%%%%%%%%%%%%%%%%%%%%%%%%%%%%%%%%%%%%%%%%%%%%%%%%%%%%%%%%%%%%%%%%%%%%%%%%%%%%%%

\section*{ACKNOWLEDGMENT}

Dr. Quinlan has certainly been helpful.

What credits do we need to give to OSM?
\commentp{Just citing it is good enough.}

%%%%%%%%%%%%%%%%%%%%%%%%%%%%%%%%%%%%%%%%%%%%%%%%%%%%%%%%%%%%%%%%%%%%%%%%%%%%%%%%

% Notes.

%possible background/links

%http://www.caliper.com/transmodeler/default.htm
%http://ieeexplore.ieee.org/stamp/stamp.jsp?tp=&arnumber=6082793
%https://www.google.com/search?sourceid=chrome&ie=UTF-8&q=Agent-Based+Traffic+Simulation+and+Traffic+Signal+Timing+Optimization+with+GPU

%%%%%%%%%%%%%%%%%%%%%%%%%%%%%%%%%%%%%%%%%%%%%%%%%%%%%%%%%%%%%%%%%%%%%%%%%%%%%%%%

\bibliographystyle{IEEEtran}
\bibliography{root}

\end{document}
